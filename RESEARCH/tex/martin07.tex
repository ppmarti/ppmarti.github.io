%%%\def\filename{martinew.tex} 
%%% THIS IS JUST MACROS
%{{{ %reminder junk
%\usepackage{epic} \usepackage{eepic} \usepackage{latexsym}
%%\usepackage{amssymb} 
%%\renewcommand{\baselinestretch}{1.5}       % 2=Double spacing  
%%%\oddsidemargin1cm \textwidth15cm \textheight25cm \topmargin-1in  
%%%% The following line for 12pt 
%%%\oddsidemargin.51cm \textwidth15cm 
%%%\textheight25cm  \topmargin-1.5cm %cf.kyoto offset -1cm!!! %\def\ninrm{}
%}}}
%{{{ global macro glossary making
%%% Global Macro glossary making macros
%%%
%%% if do not require glossary use second following line 
%\newcommand{\noglossaryignore}[1]{#1}
\providecommand{\noglossaryignore}[1]{}
%%%N.B. correct practice is to leave this as it is (!) and insert a 
%%%     copy of the third line up from here (uncommented) 
%%%     into your document if you want a glossary. 
%
%%% create glossary entry in format   MACRONAME   OUTPUT:
\newcommand{\globalglossaryentry}[3]{\makebox[1.5in][l]{\tt $\backslash${#1}} 
\makebox[1.1in][l]{{$#2$}} \makebox[2.5in][l]{{#3}}\newline} 
%
%%% metacommand to creat newcommand (abbreviation) and glossary entry data:
\newcommand{\newcommandabbreviation}[3]{\newcommand{#1}{#2}%
\noglossaryignore{\globalglossaryentry{#3}{#2}{}}}
%
%%% metacommand to creat renewcommand (abbreviation) and glossary entry data:
\newcommand{\renewcommandabbreviation}[3]{\renewcommand{#1}{#2}%
\noglossaryignore{\globalglossaryentry{#3}{#2}{}}}
%
%%% as above but for general macro:
\newcommand{\newcommandmacro}[4]{\newcommand{#1}{#2}%
\noglossaryignore{\globalglossaryentry{#3}{#2}{#4}}}
\newcommand{\gge}[3]{\noglossaryignore{\globalglossaryentry{#1}{#2}{#3}}}
%
%{{{ make glossary macros (local version, not used here)
%%%% HERE we make the glossary stuff.
%%%%
%%%% First the header for the glossary:
%%%\newcommand{\glossaryhead}{\[ \] {\bf Glossary} \newline}
%%%%
%%%% Next the glossary entries (the format is 
%%%% notation --- description --- TeX command name --- page):
%%%\newcommand{\glossaryentry}[5]%
%%%{\makebox[1in][l]{{$#2#5$}} \makebox[3.5in][l]{{#3}} 
%%% \makebox[1in][l]{{\tt $\backslash$#4 }}\pageref{#4}\newline}
%%%%
%%%% Next we create the command which simultaneously creates 
%%%% the TeX command and specifies its glossary entry:
%%%\newcommand{\metacommand}[5]{\newcommand{#1}{#2} 
%%%\glossaryentry{#1}{#2}{#3}{#4}{#5}}
%%%%
%%%% NOTE that there must be a \label{command_name} in the text 
%%%% where the notation is defined. 
%%%%
%}}}
%\glossaryhead
%\metacommand{\Set}{{\bf Set}}{Category of sets}{Set}{} 
%
%
%}}}
%{{{ global macro defns
%{{{ addresses (city etc)
\newcommand{\myaddress}%
{\parbox{3in}{\footnotesize \begin{center} 
Mathematics Department, City University, \\  
Northampton Square, London EC1V 0HB, UK.\end{center}}}
%}}}
%{{{ xfig patches
%\def{\SetFigFont{#1}{#2}{#3}}{}
\newcommand{\twlrm}{} \def\tenrm{} \def\ninrm{} \def\sixrm{} \def\sevrm{}
%}}}
%{{{ minidef and minicapt
\newcounter{minidef}[section]
\renewcommand{\theminidef}{\thesection.\arabic{minidef}}
\newcommand{\mdef}{\refstepcounter{minidef} 
\medskip \noindent ({\bf \theminidef}) }

%{{{ old style macros
\newcommand{\mystufffont}{\textsc} %% this sets up font for 
                                   %% prop/thm/etc label.
\newcommand{\mystart}{\medskip \noindent}  
                                   %% this sets up start behaviour
                                   %% for prop/def/etc
\newcommand{\mygap}{\smallskip \newline \noindent}  
                                   %% this sets up end behaviour
                                   %% for prop/def/etc
%}}}
%{{{ newtheoremstyles: pu, puu
%\swapnumbers
\newtheoremstyle{pu}% name
{7pt}%
{7pt}%
{\it}% body font
{}% indent amount
{}% {\textsc}% headfont
{.}% puntuation
{ }% gap
{\thmnumber{({\bf #2}) }\thmname{\textsc{#1}}\thmnote{#3}}% headspec
\newtheoremstyle{puu}% name
{3pt}%
{3pt}%
{\rm}% body font
{}% indent amount
{}% {\textsc}% headfont
{.}% puntuation
{ }% gap
{\thmnumber{({\bf #2}) }\thmname{\textsc{#1}}\thmnote{#3}}% headspec

%}}}
%{{{ old minidef-style stuff
%\newcommand{\mex}{\refstepcounter{minidef} 
%\medskip \noindent ({\bf \theminidef}) {\mystufffont{Example.}} }
%\newcommand{\mrem}{\refstepcounter{minidef} 
%\medskip \noindent ({\bf \theminidef}) {\mystufffont{Remark.}} }
%\newcommand{\mpr}[1]{\refstepcounter{minidef} 
%\mystart %\medskip \noindent 
%({\bf \theminidef}) {\mystufffont{Proposition.}}
%{\em #1} \mygap}
%\newcommand{\mde}[1]{\refstepcounter{minidef} 
%\mystart %\medskip \newline \noindent 
%({\bf \theminidef}) {\mystufffont{Definition.}}
%{\em #1} %\smallskip \newline \noindent 
%\mygap}

%}}}
\theoremstyle{pu}
\newtheorem{mmpr}[minidef]{Proposition}
\newcommand{\mpr}[1]{\begin{mmpr} #1 \end{mmpr}}

\theoremstyle{puu}
\newtheorem{mde}[minidef]{Definition}
\newtheorem{mex}[minidef]{Example}
\newtheorem{mrem}[minidef]{Remark}

\newcommand{\muex}{\noindent{\mystufffont{Example.}} }
\newcommand{\mupr}{\smallskip\noindent{\mystufffont{Proposition.}} }

\newcounter{minicapt}
\newenvironment{capt}[1]
{\refstepcounter{minicapt}\stepcounter{figure}
\begin{center}\sf Figure \theminicapt. }{ \end{center}}
%}}}
%{{{ theorems
\newtheorem{theo}{Theorem}      
\newtheorem{cor}{Corollary}[theo]   
\newtheorem{de}[minidef]{Definition}     
\newtheorem{pr}[minidef]{Proposition} 
%\newtheorem{co}{Corollary}[pr]  
\newtheorem{co}[minidef]{Corollary} 
\newtheorem{rem}[minidef]{Remark} 
\newtheorem{lem}{Lemma} 
\newtheorem{claim}[minidef]{Claim}
\newtheorem{example}[minidef]{Example}
%}}}
%{{{ greek and other abbreviations
\noglossaryignore{GREEK ETC.\newline}
\newcommandabbreviation{\e}{\epsilon}{e}        
\newcommandabbreviation{\lam}{\lambda}{lam}  
\newcommandabbreviation{\la}{\langle}{la}        
\newcommandabbreviation{\ran}{\rangle}{ran}
\newcommandabbreviation{\ha}{\#}{ha}             
\newcommandabbreviation{\rmap}{\rightarrow}{rmap}
\newcommandabbreviation{\aaa}{\alpha}{aaa}        
\newcommandabbreviation{\ab}{\alpha,\beta}{ab}
\newcommandabbreviation{\aab}{a(\ab )}{aab}       
%}}}
%{{{ rings
\noglossaryignore{\newline RINGS\newline}
%\newcommand{\C}{{\bf C} \!\!\!\! I}        % Complex numbers
\newcommandabbreviation{\HH}{H \!\!\! I}{HH}               % Hecke algebra
%\newcommand{\Z}{Z \!\!\! Z}                % Integers
%\newcommand{\N}{N \!\!\!\!\! I \;}         % Natural numbers
\newcommandabbreviation{\C}{\mathbb C}{C}
\newcommandabbreviation{\N}{\mathbb N}{N}   %new AMS versions (was \Bbb)!
\newcommandabbreviation{\Z}{\mathbb Z}{Z}      % AMS versions!!!!!!
\renewcommandabbreviation{\Re}{\mathbb R}{Re}
\newcommandabbreviation{\R}{{\mathbb R}}{R}
% NB, above seems to clash with prosper class, PPM.
\newcommandabbreviation{\Q}{\mathbb Q }{Q}
\renewcommandabbreviation{\H}{\mathbb H }{H}
%}}}
%{{{ symmetric group
\noglossaryignore{\newline SYMMETRIC GROUP\newline}
\def\Sym(#1){\Sigma(#1)}                   % generic symbol for symmetric group
\gge{Sym(-)}{\Sym(-)}{symmetric group on - objects}
\def\Sy(#1){\Sigma_{#1}}                   % symmetric group irrep.
\gge{Sy(-)}{\Sy(-)}{symmetric group irreducible -}
\def\sym(#1){\mbox{\LARGE s}(#1)}        % another generic symbol for
 % sym gp
\gge{sym(-)}{\sym(-)}{symmetric group on - objects (variant)}
\def\sy(#1){\mbox{\LARGE s}({#1})}        % another symm gp irrep. 
\gge{sy(-)}{\sy(-)}{symmetric group irreducible - (variant)}
\newcommandmacro{\cs}{\C \, \sy(n)}{cs}{symmetric group algebra over $\C$}
%}}}
%{{{ partitions/sets
\noglossaryignore{\newline PARTITIONS/SETS\newline}
%\newcommandmacro{\Nset}{\underline{n}}{Nset}{set of natural numbers to n}
\newcommand{\Nset}[1]{\underline{#1}}
\gge{Nset\{-\}}{\Nset{-}}{set of natural numbers to -} %%% (default n)}
\def\nset(#1){ \{ #1 \}_{ \underline{n} }} % the set {#1}_{n}
\gge{nset(-)}{\nset(-)}{a set $-\times\Nset$}
\def\ul(#1){_{\underline{#1}}}             % _underline #1
\gge{ul(-)}{{}\ul(-)}{subscript underline -}
\def\Ee(#1){{\bf E}_{#1}}                  % set of equiv. relations
\gge{Ee(-)}{\Ee(-)}{set of equivalence relations on set -}
\def\Eee(#1){{\bf E}_{\{ #1 \}_{\underline{n}}}}   %ditto for nset
\gge{Eee(-)}{\Eee(-)}{ditto for nset}
\def\Een(#1,#2){{\bf E}_{\{ #1 \}_{\underline{#2}}}}   %ditto for n+1set
\def\Ssn(#1,#2){{\bf S}_{\{ #1 \}_{\underline{#2}}}}   %partitions for n+1set
\def\Ss(#1){{\bf S}_{#1}}                  % set of partitions
\def\Sss(#1){{\bf S}_{\{ #1 \}_{\underline{n}}}}   %ditto for nset
\def\bbc(#1){((\beta_1)(\beta_2)...(\beta_{#1}))}      % beta singletons 
\newcommandmacro{\Ln}{{\Gamma}^{n}}{Ln}{large index set}
\newcommandmacro{\LnQ}{{\Gamma}^{n}_Q}{LnQ}{index set}
\newcommandmacro{\Zz}{\zeta}{Zz}{`shape' function}
\newcommand{\Ww}[1]{\mbox{\LARGE $w$}_{#1}}         % funny set
%}}}
%{{{ partition algebra
\noglossaryignore{\newline PARTITION ALGEBRA\newline}
\def\ka(#1){\kappa_{#1}}                   % maps AP_{n=#1}A to P_{n-1}
\def\Sm(#1){\Sigma_{#1}}                   % image of S_lamda in P_n
\newcommandmacro{\com}{\bullet}{com}{bullet composition}
%%%ELEMENTS%%%
\newcommandmacro{\enm}{\; e^n(\! m\! ) \;}{enm}{product of idempotents}
\def\Ai(#1){ A^{ #1 \cdot } }              % A_i
\def\Aij(#1,#2){ A^{ #1  #2 } }            % A_ij
\newcommandmacro{\One}{\mbox{\bf $1 \!\!\! 1$}}{One}{algebra unit 1}
\newcommand{\Ff}[2]{F_{#1}^{(#2)}}       % unnormalized idempotent
\newcommand{\ef}[2]{\prod_{j=1}^{#2-#1}e_j} % E idempotent
\newcommand{\efbr}[2]{\left(\prod_{j=1}^{#2-#1}e_j\right)} % (E idempotent)
\newcommand{\Ef}[2]{E_{#1}^{(#2)}}          % notation for E idempotent
\def\fu{
\left( \prod_{i=1}^{n} \left( \One - \Aij(i, \; n+1) \right) \right)
}                                          % chi idempotent 
\def\Fu{\mbox{\Large $\chi$}}              % notation for above idempotent
%%%MODULES%%%
\newcommandmacro{\Bp}{B_p}{Bp}{partition basis}
\def\Bb(#1){B_p[#1]}                       % partition basis
\def\Pp(#1){P_n[#1]}                       % left module
\def\Ps(#1){P_n[#1] \! /}                  % left module
\newcommandmacro{\Ph}{\hat{P}}{Ph}{P hat  algebra}
\def\Is(#1){\sim^{#1}}                     % is equivalent under a to
%}}}
%{{{ modules
\noglossaryignore{\newline MODULES\newline}
\def\Wm(#1){{\cal S}_{#1}}                 % Weyl module S 
\gge{Wm(-)}{\Wm(-)}{Weyl module with index -}
\def\wm(#1,#2){{}_{#1}{\cal S}_{#2}}       % Weyl module nS
\gge{wm(-1,-)}{\wm(-1,-)}{Weyl module with index -}
\def\Ind(#1,#2,#3){\mbox{Ind}_{#1}^{#2}#3} % induction
\gge{Ind(-1,-2,-)}{\Ind(-1,-2,-)}{induction}
\def\Res(#1,#2,#3){\mbox{Res}_{#1}^{#2}#3} % restriction
\gge{Res(-1,-2,-)}{\Res(-1,-2,-)}{restriction}
\newcommandabbreviation{\weyl}{standard}{weyl}
\newcommandabbreviation{\mod}{\mbox{mod}}{mod}
\newcommandabbreviation{\head}{\mbox{head }}{head}
\newcommandabbreviation{\Weyl}{Weyl}{Weyl}
\def\SS(#1){{\cal S}_{#1}}                 % Specht/Weyl module
\gge{SS(-)}{\SS(-)}{Specht/Weyl module index -}
\def\LL(#1){{\cal L}_{#1}}                 % Simple module
\gge{LL(-)}{\LL(-)}{simple module index -}
%}}}
%{{{ functors/maps
\noglossaryignore{\newline FUNCTORS/MAPS\newline}
\newcommandmacro{\Gg}{{\cal G}}{Gg}{G Functor}
\newcommandmacro{\Fg}{{\cal F}}{Fg}{F Functor}
\newcommandmacro{\ra}{\rightarrow}{ra}{}
\def\ses(#1,#2,#3){0\ra #1 \ra #2 \ra #3 \ra 0}   %short exact sequence
\gge{ses(1,2,-)}{\ses(1,2,-)}{\hspace{.5in} short exact sequence}
\def\starr(#1){ \stackrel{ #1 }{\longrightarrow} }
\gge{starr(-)}{\starr(-)}{}
\newcommandmacro{\doublerightarrow}{\; -\!\!\! -\!\!\!\!\!\! \gg \;}
{doublerightarrow}{}%{ $---->>$ }
%}}}
%{{{ partition algebra maps
\noglossaryignore{\newline PARTITION ALGEBRA MAPS\newline}
\newcommandmacro{\smap}{s}{smap}{`inclusion' map}
\newcommandmacro{\tmap}{t}{tmap}{$ P_n -> S_n$}
\newcommandmacro{\pmap}{\psi}{pmap}{$ S_n -> P_n $}
%}}}
%{{{ miscellaneous
\noglossaryignore{\newline MISC.\newline}
\def\Amap(#1){{\cal A}_{#1}}               % variant inclusion A_Gamma
\gge{Amap(-)}{\Amap(-)}{}
\def\Rr(#1){R_{#1}}                        % restriction of E
\gge{Rr(-)}{\Rr(-)}{restriction of E}
\def\Cr(#1){C_{#1}}                        % restriction of E to N
\gge{Cr(-)}{\Cr(-)}{restriction of E to N}
\newcommandmacro{\Tm}{{\cal T}}{Tm}{Transfer Matrix}
%used!%\newcommand{\Gg}{{\cal G}}                 % graph
\def\On(#1){{\cal I}_{#1}}
\gge{On(-)}{\On(-)}{}
\newcommandmacro{\UU}{\underline{\sqcup}}{UU}{}  
\newcommandmacro{\UUU}{\sqcup}{UUU}{}  
\newcommandmacro{\Vq}{V_Q^{\otimes n}}{Vq}{Potts config. space}
\def\bs(#1,#2){\mbox{{\Large $\ast$}}^{#1}_{#2}}  % general plumbing multiplier
\gge{bs(-,-)}{\bs(-,-)}{general plumbing multiplier}
%\def\ddots{{}^.._.}
\newcommand{\ignore}[1]{}
\gge{ignore\{-\}}{\ignore{-}}{ignore argument!}
%}}}
%{{{ math stuff
\def\choo(#1,#2){ \left( \begin{array}{c} #1 \\ #2 \end{array} \right) } %choose
\gge{choo(-1,-)}{\choo(-1,-)}{choose}
%%\newcommand{\choose}[2]{\mbox{% 
%%\left( \!\! \begin{array}{c} #1 \\ #2 \end{array} \!\! \right)}}
%\newcommandmacro{\Qed}{$\Box$}{Qed}{QED}
\newcommand{\Qed}{$\Box$}%{Qed}{QED}
\gge{Qed}{\mbox{\Qed}}{QED}
\def\staq(#1){\stackrel{#1}{=}}            % stack =
\gge{staq(-)}{\staq(-)}{}
\def\stam(#1){\stackrel{#1}{\rightarrow}}  % stack ->
\gge{stam(-)}{\stam(-)}{}
\def\mat{ \left( \begin{array} }    
\def\tam{ \end{array}  \right) }
\gge{mat/tam}{...}{matrix delimiters}
%}}}
%{{{ equation environments etc (draft/final)
%%%%%%%%%%%%
%ENVIRONMENTS:   %%%FINAL VERSION VERSION (NOT DRAFT)
%%%%%%%%%%%%
\newcommand{\beq}{\begin{equation} }
\def\eql(#1){ \begin{equation} \label{#1} 
%
%\hspace*{-108pt} {}_{eq.(#1)} \; \hspace{72pt} \; 
}
\newcommand{\eq}{\end{equation} }
\def\eqal(#1){\begin{eqnarray} \label{#1} }
\def\eqa{\end{eqnarray} }
\def\lab(#1){\label{#1}
%
%${}_{lab.(#1)} \;$
}
\def\prl(#1){ \begin{pr} \label{#1} 
%
%${}_{pr.(#1)} \;  \; $
}
\def\del(#1){ \begin{de} \label{#1} 
%${}_{pr.(#1)} \;  \; $                 %(hash out in final version)
}
\newcommand{\smeq}[1]{\[ \mbox{\small $ #1 $} \] }    %small eqn
\gge{smeq\{-\}}{...}{small equation}
\newcommand{\fneq}[1]{\[ \mbox{\footnotesize $ #1 $} \] }   %vsmall eqn
\gge{fneq\{-\}}{...}{very small equation}
%}}}
%\newcommand{\Ln}{{\cal L}_{n}}           % index set (! mult defined)
%}}}
%{{{     local defns(Hecke/blob)
\noglossaryignore{\newline HECKE/BLOB\newline}
\newcommandmacro{\Hnq}{H_n(q)}{Hnq}{ * freestanding symbol}
\newcommandmacro{\Hn}{H_n}{Hn}{      *-mod etc.}
\newcommandmacro{\A}{{\cal A}}{A}{}
\newcommandmacro{\Cwts}{C}{Cwts}{}
\newcommandmacro{\CA}{{\cal A}}{CA}{}
\def\words#1{\mbox{ #1}}
\newcommandmacro{\calA}{{\cal A}}{calA}{}
\newcommandmacro{\modi}{\mbox{Mod} }{modi}{was mod not modi!}
\newcommandmacro{\Wgen}{{\Bbb S}}{Wgen}{}
\def\ol(#1){\overline{#1}}
\newcommandmacro{\st}{\mbox{St}}{st}{}
\def\twiddle{\~}   %  globally replaced by \tilde !!!!
\def\CMult(#1,#2){(#1:#2)}
\def\CM(#1,#2){( #1 : #2 )}
\def\FMult#1,#2{(#1:#2)}
\def\CF#1,#2{(#1:#2)}
\def\RMod#1{#1-\!\mbox{mod}}
\def\LMod#1{\mbox{mod}\!-#1}
\newcommandmacro{\Top}{\mbox{Top}}{Top}{}
\newcommandmacro{\Soc}{\mbox{Soc}}{Soc}{}
\newcommandmacro{\Head}{\mbox{Head}}{Head}{}
\newcommandmacro{\Filt}{{\cal F}}{Filt}{}
\newcommandmacro{\Mod}{\mbox{mod}}{Mod}{}
\newcommandmacro{\Resi}{\mbox{Res }}{Resi}{was without i!}
\newcommandmacro{\Indi}{\mbox{Ind }}{Indi}{was without i!}
\newcommand{\II}[2]{I^{#1}_{#2}}   %projection after induction (after projection)
\def\RR(#1,#2){R^{#1}_{#2}}   %projection after restriction (after projection)
\def\TT(#1,#2){T^{#1}_{#2}}   %translation
\def\implies{\Rightarrow}
\def\qSchur{S_q(n,N)}
\def\Uq{U_q(sl_N)}
\def\Usl{U(sl_N)}
\def\qGroup{$q$-group}
\def\VV{V_N^{\otimes n}}
\def\Hom{\mbox{Hom}}
%\def\End{\mbox{End}}
\def\v{v}                  % soergel parameter
\def\len{\mbox{len}}       % length order
\def\bigplus{\mbox{\LARGE $+$}}
\def\dom{dominance }
\def\ldom{\unlhd}
\def\gdom{\unrhd}
\def\laffine{$l$-affine }
\newcommand{\nt}[1]{{\em #1}}
\def\Sp{\Delta}           % Specht module
\def\Stan{\Delta}         % Standard module
\def\Ysym{Y}
\def\eqn#1{equation (\ref{#1}) }
\newcommand{\eqr}[1]{equation~(\ref{#1})}
\def\HP#1{P'_{#1}}
\def\UP#1{P_{#1}}
\def\HSp#1{\Sp'_{#1}}
\def\USt#1{\Stan_{#1}}
\def\HL#1{L'_{#1}}
\def\UL#1{L_{#1}}
\def\Chi{\chi}
\def\Cartan{C}
\def\Rad{\mbox{Rad }}
\def\facets#1{{\cal A}_{#1}}
\def\St{\st} % usage \St(...)
%
%}}}
%{{{     local defns(partition)
\newcommandmacro{\Ann}{\mbox{Ann}}{Ann}{}
\newcommandmacro{\Cen}{\mbox{Cen}}{Cen}{}
\newcommandmacro{\End}{\mbox{End}}{End}{}
%used already!%\newcommand{\Pp}{{\cal P}}
\newcommandabbreviation{\semisimple}{semisimple}{semisimple}
\newcommandabbreviation{\Bratteli}{Bratteli}{Bratteli}
\newcommandabbreviation{\JBC}{Jones Basic Construction}{JBC}
\newcommandabbreviation{\pa}{partition algebra}{pa}
\newcommandabbreviation{\TM}{transfer matrix}{TM}
\newcommandabbreviation{\PM}{Potts model}{PM}
\newcommandabbreviation{\QSC}{quantum spin chain}{QSC}
\newcommandabbreviation{\Hamiltonian}{Hamiltonian}{Hamiltonian}
\newcommandabbreviation{\YS}{Young symmetrizer}{YS}
%}}}
%{{{    %local defns (blob)

%\def\TLB{TL_{B}}
%\def\ATL{TL_{\hat{A}}}
%\def\MW{\cite{MartinWoodcock}}
%\def\GL{\cite{GrahamLehrer}}
%\def\sig{\sigma}
%\def\qbox#1{[#1]}
%\newcommand{\qn}{{\cal A}}%ring of q-numbers

%}}}
%{{{ Local vars
% � Local Variables

% Local Variables:
% eval: (standard-display-european 1)
% mode: LaTeX
% folded-file: t
% End:

% �
%}}}
