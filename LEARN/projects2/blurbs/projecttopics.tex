\documentstyle[a4,12pt]{article}
\begin{document}

{\bf Project Titles.}

\medskip

1. Permutation  groups (3MB, 4MM). Best done in conjunction with, or after Groups and Symmetry (3071).
The project concerns group actions on sets. It can easily become combinatorial, e.g. the study of graphs 
with strong symmetry properties (such as distance-transitive graphs). Connections can be made
to model theory (infinite permutation groups can arise as automorphisms of structures built by model theory).

P.J. Cameron, {\em Permutation groups}, Cambridge Univ. Press, 1999.

J.D. Dixon, P. Mortimore, {\em Permutation groups}, Springer, 1996

\medskip

2. Combinatorial group theory. This is the study of groups described by a collection of generators
(so every member of the group can be written as a product of a sequence of these generators) and
 relations holding among these generators. 
Early material would cover free groups (for which there are no relations), and free products of groups.
Typical later material would concern free products with amalgamation and HNN extensions, and connections
 might well be made to algebraic topology (eg if done as 4MM).
 It is best 
done either as 4MM, or in the second semester as 3083, after a student has done Groups and Symmetry (3071).

R.C. Lyndon, P.E. Schupp, {\em Combinatorial Group Theory}, Springer, 1977.

W. Magnus, A. Karrass, D. Solitar, {\em Combinatorial Group theory}, Wiley, 1966.

\medskip

2. Steiner systems. (3MB or 4MM) A $(t,k,v)$-Steiner system is a set $P$ of
 size $v$ of points, and a collection ${\cal B}$ of subsets of $P$, each of size $k$, called, {\em blocks}, such that
any $t$ points from $P$ lie in a unique block. The study of these belongs to combinatorics, but has close connections
 to group theory and coding theory. The project might explore these connections, or might aim for classification 
results, given restrictions on $t,k,v$.

P.J. Cameron, {\em Combinatorics. Topics, techniques, algorithms}, Cambridge Univ. Press 1994.

J.H. van Lint, R.M. Wilson, {\em A course in combinatorics}, Cambridge Univ. Press 1992.

\medskip

3. The $p$-adic numbers. The real numbers are obtained from the rationa numbers by `completing', i.e. adding limits of 
Cauchy sequences, with respect to 
Cauchy sequences.  For any prime $p$, there is another `metric' or distance on the rational numbers. By completing with
 respect to this, one obtains the field ${\bf Q}_p$ os $p$-adic numbers. This has strange properties: for example, all 
triangles are isosceles. The project could take various directions, for example the Hasse principle, that a quadratic form 
over ${\bf Q}$ has a rational solution if and only if it has a solution in the reals and in all fields ${\bf Q}_p$. 
Another direction for the project would be to develop the general theory of valued fields.

J.W.S. Cassels, {\em Local fields}, Cambridge Univ. Press, 1986.

P. Ribenboim, {\em The theory of classical valuations}, Springer, 1998.

\medskip

4. Hilbert's Tenth Problem. The theorem of Davis, Putnam, Robinson and Matijacevic (the final step published by
 Matijasevic in 1970)
states that if $A$ is a computably enumerable set,
that is, a set whose members could be listed  by a computer, then 
there is a polynomial $p(X,Y_1,\ldots,Y_n)$ over the integers such that
$A=\{x\in {\bf Z}: \exists y_1,\ldots y_n \in {\bf Z}: p(x,y_1,\ldots,y_n)=0\}$. 
This gives a negative solution to Hilbert's tenth problem, which asks whether, given a polynomial equation in an
 arbitrary number of variables over the 
integers, we can determine whether or not the equation is solvable in the integers. The project will be an attempt to
 describe the proof. The project is really a mixture of logic and elementary number theory.


Y. Matiiasevich, {\em Hilbert's tenth problem}, MIT Press, 1993.
\medskip

5. Ramsey Theory: This is really a topic in combinatorics, but has connections to logic (e.g. set theory), number theory, and 
other subjects. An easy (infinitary) version of the theorem says that any infinite graph contains either an infinite
 set of vertices 
any two of which are joined by an edge, or an infinite set of vertices no two of which are joined by an edge. There are 
many subtle variants of this: working with different infinite cardinals, 
statements about `monochromatic' arithmetic progressions for colourings of natural numbers, calculation of
 bounds for finite versions of Ramsey's Theorem, versions for vector space sover finite fields, 
connections to G\"odel's incompleteness.

P.J. Cameron, {\em Combinatorics. Topics, techniques, algorithms}, Cambridge Univ. Press 1994.

R.L. Graham, B.L. Rothschild, J. Spencer, {\em ramsey Theory}, Wiley, 1990.

\medskip

6. Model theory (usually 4MM)
Model theory is the area of mathematical logic which interacts most closely with other parts of mathematics (e.g.
algebra and geometry). It concerns expressibility of mathematical properties in formal logical languages.
Background from 3285 or 5285 is pretty essential for this. The project would probably give a brief resum\'e of first order 
logic and compactness,
and then pick up some topic of model theory. For example: automorphism groups of first order structures (for those who also like 
group theory); o-minimality (which can be viewed as the model theory of geometric proeprties of the real numbers); the
 general theory of model companions, with examples;
Morley's theorem (a theorem proved in 1965 about the extent to which a structure is determined by the first order
 sentences true of it); finite model theory.
 
 W. Hodges, {\em A shorter model theory}, Cambridge Univ. Press, 1997.

\medskip

7. Set theory (4MM)
This is best done by someone who has taken the Set Theory module 3123 and done some other logic modules. The 
main goal would probably be to describe G\"odel's
 Constructible Universe 
$L$,
and explain why it satisfies the Axiom of Choice (AC) and the Continuum Hypothesis (CH). This yields that if the
 other standard  axioms (ZF) of set theory are consistent, so is ZFC+AC+CH.
 
K. Kunen, {\em Set theory. An introduction to independence proofs}, North-Holland, 1980. 
 
 \medskip
 
 8. Coding theory (3MB, 4MM)
 This concerns ways of sending a message (eg a sequence of zeros and ones) through space in such a way that, even if errors 
occur in
 transmission, there is a good chance of decoding correctly. The subject is highly combinatorial, but also involves linear
 algebra. 
 Topics covered would probably begin with  sphere packing, generator and parity check matrices, Hamming codes, and might 
include some of:
 self-dual codes,  cyclic codes,
 Reed-Muller codes, weight enumerators.
 
 R. Hill, {\em A first course in coding theory}, Oxford Univ. Press, 1991.
 
 J.H. van Lint, {\em Introduction to coding theory}, Springer, 1982.
    
 \end{document}